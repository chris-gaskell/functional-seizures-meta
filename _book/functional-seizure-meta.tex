% Options for packages loaded elsewhere
\PassOptionsToPackage{unicode}{hyperref}
\PassOptionsToPackage{hyphens}{url}
%
\documentclass[
  12pt,
  openany]{book}
\title{The effectiveness of psychological therapy for treatment of functional seizures in adults}
\usepackage{etoolbox}
\makeatletter
\providecommand{\subtitle}[1]{% add subtitle to \maketitle
  \apptocmd{\@title}{\par {\large #1 \par}}{}{}
}
\makeatother
\subtitle{A systematic review and meta-analysis}
\author{Dr.~Chris Gaskell \and Dr.~Niall Power \and Dr.~Gregg Rawlings \and Dr.~Barbora Novakova \and Dr.~Mel Simmonds-Buckley \and Dr.~Stephen Kellett \and Prof.~Markus Reuber}
\date{2022-04-18}

\usepackage{amsmath,amssymb}
\usepackage{lmodern}
\usepackage{iftex}
\ifPDFTeX
  \usepackage[T1]{fontenc}
  \usepackage[utf8]{inputenc}
  \usepackage{textcomp} % provide euro and other symbols
\else % if luatex or xetex
  \usepackage{unicode-math}
  \defaultfontfeatures{Scale=MatchLowercase}
  \defaultfontfeatures[\rmfamily]{Ligatures=TeX,Scale=1}
\fi
% Use upquote if available, for straight quotes in verbatim environments
\IfFileExists{upquote.sty}{\usepackage{upquote}}{}
\IfFileExists{microtype.sty}{% use microtype if available
  \usepackage[]{microtype}
  \UseMicrotypeSet[protrusion]{basicmath} % disable protrusion for tt fonts
}{}
\usepackage{xcolor}
\IfFileExists{xurl.sty}{\usepackage{xurl}}{} % add URL line breaks if available
\IfFileExists{bookmark.sty}{\usepackage{bookmark}}{\usepackage{hyperref}}
\hypersetup{
  pdftitle={The effectiveness of psychological therapy for treatment of functional seizures in adults},
  pdfauthor={Dr.~Chris Gaskell; Dr.~Niall Power; Dr.~Gregg Rawlings; Dr.~Barbora Novakova; Dr.~Mel Simmonds-Buckley; Dr.~Stephen Kellett; Prof.~Markus Reuber},
  hidelinks,
  pdfcreator={LaTeX via pandoc}}
\urlstyle{same} % disable monospaced font for URLs
\usepackage[left=2cm, right=2cm, top=2cm, bottom=2cm]{geometry}
\usepackage{longtable,booktabs,array}
\usepackage{calc} % for calculating minipage widths
% Correct order of tables after \paragraph or \subparagraph
\usepackage{etoolbox}
\makeatletter
\patchcmd\longtable{\par}{\if@noskipsec\mbox{}\fi\par}{}{}
\makeatother
% Allow footnotes in longtable head/foot
\IfFileExists{footnotehyper.sty}{\usepackage{footnotehyper}}{\usepackage{footnote}}
\makesavenoteenv{longtable}
\usepackage{graphicx}
\makeatletter
\def\maxwidth{\ifdim\Gin@nat@width>\linewidth\linewidth\else\Gin@nat@width\fi}
\def\maxheight{\ifdim\Gin@nat@height>\textheight\textheight\else\Gin@nat@height\fi}
\makeatother
% Scale images if necessary, so that they will not overflow the page
% margins by default, and it is still possible to overwrite the defaults
% using explicit options in \includegraphics[width, height, ...]{}
\setkeys{Gin}{width=\maxwidth,height=\maxheight,keepaspectratio}
% Set default figure placement to htbp
\makeatletter
\def\fps@figure{htbp}
\makeatother
\setlength{\emergencystretch}{3em} % prevent overfull lines
\providecommand{\tightlist}{%
  \setlength{\itemsep}{0pt}\setlength{\parskip}{0pt}}
\setcounter{secnumdepth}{5}
\usepackage{booktabs}
\usepackage{amsthm}
\makeatletter
\def\thm@space@setup{%
  \thm@preskip=8pt plus 2pt minus 4pt
  \thm@postskip=\thm@preskip
}
\makeatother
\usepackage{booktabs}
\usepackage{longtable}
\usepackage{array}
\usepackage{multirow}
\usepackage{wrapfig}
\usepackage{float}
\usepackage{colortbl}
\usepackage{pdflscape}
\usepackage{tabu}
\usepackage{threeparttable}
\usepackage{threeparttablex}
\usepackage[normalem]{ulem}
\usepackage{makecell}
\usepackage{xcolor}
\ifLuaTeX
  \usepackage{selnolig}  % disable illegal ligatures
\fi
\usepackage[]{natbib}
\bibliographystyle{apalike}

\begin{document}
\maketitle

{
\setcounter{tocdepth}{1}
\tableofcontents
}
\hypertarget{Abstract}{%
\chapter*{Abstract}\label{Abstract}}
\addcontentsline{toc}{chapter}{Abstract}

\textbf{Objective:}

\textbf{Method:}

\textbf{Results:}

\textbf{Conclusion:}

\textbf{Keywords:}

\textbf{Clinical Significance of this Article:}

Introduction
Functional seizures (FS) are episodes characterized by alterations of awareness, self-control and/or perception (Reuber \& Rawlings, 2017; Brown \& Reuber, 2016). FS are known by a range of different terms, such as non-epileptic attack disorder, psychogenic nonepileptic seizures and dissociative seizures (Rawlings \& Reuber, 2018). FS are considered as being out of the patient's own control and thereby distinct from factitious or malingered seizures (Brown \& Reuber, 2016).
FS can superficially resemble epileptic seizures but are not associated with epileptiform activity. Instead, FS have been conceptualized as an automatic response to internal or external cues that are perceived as adverse, distressing, or threatening (Brown \& Reuber, 2016). FS are more common in certain groups; for example, in females, those from a lower socioeconomic status (Goldstein et al., 2020) and people with a learning disability (Rawlings et al., 2021). The prevalence of FS has been estimated as 50/100.000 per year (Kanemoto et al., 2017). Together with syncope and epilepsy FS are one of the three common causes of transient loss of consciousness (TLOC). As such, they account for up to 20\% of referrals to outpatient seizure clinics (Angus-Leppan et al., 2008). In the current nosologies, FS are classified as a psychiatric disorder (ICD and DSM REF). However, they have been named an ``orphan'' disorder (Baslet, 2012), with patients at risk of falling between different specialities and a lack of clarity about who is responsible for treating people with this diagnosis (Rawlings et al., 2018).
Individuals with FS are at a higher risk of experiencing axis I (Brown and Reuber, 2016) and axis II disorders (Reuber et al., 2004), in addition to psychosocial and environmental difficulties (axis IV) (Robson et al., 2018) and they report lower health-related quality of life (axis V) (Jones, Reuber \& Norman 2016). Unfortunately, the prognosis seems to be poor for a substantial proportion of people with FS (Reuber et al., 2003). However, predictors of a positive outcome (i.e., seizure cessation) include prompt and supportive delivery of the diagnosis, collaboration, mental and physical health review (including discontinuation of any of anti-seizure medications provided for a previous erroneous diagnosis of epilepsy), and engagement in a suitable form of psychological treatment (REF).\\
Psychological therapy is considered the intervention of choice for FS (Goldstein et al., 2004). Specific forms of psychological intervention which have been investigated (to vary extents) for their effectiveness with functional seizures includes, cognitive-behavioural therapy, psychoanalytic psychotherapy, mindfulness, eye movement desensitization and reprocessing (EMDR), family therapy, motivational interviewing, psychoeducation, and hypnosis. Despite this, there remain many uncertainties regarding treatment (Rawlings \& Reuber, 2018). These include questions about the effectiveness, efficacy, acceptability, mechanisms of change; and then also what is the optimal: therapeutic modality, duration, and outcome measure. A systematic review and meta-analysis were conducted in September 2016 examining the evidence for effectiveness of psychological interventions in treating FS. In total, 13 studies were found to be eligible at the time, representing data from 228 participants with FS. The authors focused on seizure frequency as the primary outcome, demonstrating that just under half (47\%) were seizure free at the end of treatment; moreover, over 80\% of those who completed treatment had experienced a seizure reduction of at least 50\% (Carlson \& Perry, 2016).
Although psychological treatments for FS have, therefore, been the subject of a meta-analysis previously, it is important to recognise the changing landscape in this research area since the review by Carlson and Perry (2017) was conducted. The largest randomised controlled trial to date examining psychological therapy for FS was published in 2020 and involved 368 patients with FS (Goldstein et al., 2020). What is more, there is growing debate regarding whether seizure frequency is the most appropriate outcome when evaluating treatments, not least because psychological factors such as depression and anxiety have been found to be a stronger predictor of health-related quality of life (HRQoL) in people with FS than seizure frequency and may potentially be more amenable to change in treatment (Rawlings et al., 2017). Other outcome domains such as HRQoL, psychosocial or occupational functioning have also been proposed as relevant to capture in assessment and treatment of patients with functional neurological disorders, including FS (Pick et al., 2020)
Review questions:
For adults accessing psychological interventions for functional seizures:
1. What is the pooled treatment effectiveness for outcome studies exploring psychological interventions for change in seizure specific outcomes (i.e., frequency, intensity, freedom).
2. Are their differences in rates of effectiveness across non-seizure related outcome domains (HR-QoL, psychological distress, depression, anxiety, functioning, dissociation etc.)?
3. Are there characteristics of patients, treatments or study design that moderate the relationship between psychological intervention and treatment outcomes?
a. For identified moderator variables which have a body of evidence from the broader psychotherapy literature, it will be hypothesized that these findings remain consistent for functional seizures (i.e., confirmatory variables).
b. For moderator variables without a body of evidence, there will be no significant difference between subgroups (i.e., exploratory variables).
4. Are changes made during treatment durable? (i.e., are improvements following the acute stage of treatment maintained at follow-up).

\hypertarget{literature}{%
\chapter{Literature}\label{literature}}

Here is a review of existing methods.

Systematic search
Four electronic databases (CINAHL, PsycINFO, MEDLINE, Cochrane Reviews) will be searched for relevant articles using pre-identified search terms. Search term combinations will include both a diagnosis and treatment term. Diagnosis terms will include functional seizures and common diagnostic equivalents (e.g., non-epileptic attack disorder, psychogenic nonepileptic seizures, dissociative seizures, conversion seizures, pseudoseizures). Broader diagnostic terms which do not necessarily confirm seizure involvement (e.g., conversion disorder, psychosomatic illness, functional neurological disorder) will not be included. For treatment, the terms psychotherapy, psychological therapy, and psychological treatment will be used (see Table 1).
Table 1:
Systematic review key terms.
Concept 1 AND Concept 2
Theme FS Treatment
Key words Functional seizures OR psychogenic non epileptic seizures OR dissociative seizures OR pseudoseizures OR non epileptic attack disorder OR non epileptic seizures OR psychogenic seizure OR nonepileptic OR PNES OR NES OR non-epileptic OR NEAD Psychotherapy OR psychological therapy OR psychological treatment OR psychological intervention
Search limiters will include English publications and adult samples, excluding reviews. The systematic search will range from 2000 to the search date to include recent articles. All publication types will be included to reduce upward bias of studies from peer-reviewed journals which tend to show (on average) larger effect-sizes (Polanin et al., 2016; Chow \& Ekholm 2018) To identify studies not captured by the electronic databases search we will also: (a) undertake forward/backward citation searching ; (b) contact leading researchers in the field; and (c) scan reference lists from key relevant reviews and meta-analyses (Carlson \& Perry, 2017). Article titles relevant to the current review will be reviewed using study exclusion/inclusion criteria.
Table 2:
Inclusion and exclusion criteria
Inclusion Exclusion
Population Adults (16 years or older) reported to have a diagnosis of FS Samples with a majority proportion of patients who are (i) under the age of 16; or (ii) experience mixed seizure disorder (FS and epilepsy).
Intervention Psychological treatment such as CBT, psychodynamic psychotherapy, psychoeducation. Delivered on a 1:1 and groups basis Solely focusing on a non-psychological treatment.\\
Comparison Any comparison group -\\
Outcome Patient reported outcome measure utilising a standardised tool assessing psychological, emotional and/or behavioural functioning Non-patient reported outcome measure
Other Case study, single-case experimental studies
Database search results will be exported (.ris files), collapsed, stripped of duplicates, and then imported to Microsoft Excel. Manuscripts for studies identified from the systematic search will be retrieved. Articles will be screened using a pre-developed and piloted screening tool (Appendix A). For studies not available through the authors institutions' an e-mail/ResearchGate message will be sent to corresponding authors to request access (two-week response time). Reasons for exclusion will only be recorded at the full-text screening stage. Screening of abstracts and titles was performed by two independent screeners (GR \& BN) with decision conflicts being resolved through consultation. Full texts will be screened by three members of the review team in tandem with conflicts being resolved by majority rule.
Types of studies
Condition of interest:
Studies that include patients experiencing functional seizures will be included. For this review, the presence of functional seizures will be accepted by manuscript sample description -- however, method of diagnosis (i.e., VEEG) will be recorded and discussed in terms of possible implications for bias. Due to the issue of nomenclature inherent in functional seizures we will not require evidence/comment that included participants hold a specific diagnosis. Patient samples that report a majority of patients (over 50\%) have a co-morbid diagnosis of epileptic and functional seizures will be excluded.
Participants
This review focuses on treatment of functional seizures in adults/older adults. Patient samples that report a majority patients (over 50\%) as being below 16 years of age will be excluded. 16 was set as the cut-off as it is typically the point at which patients access adult health services in the UK. Children and adolescents are not included as there are likely to be differences in terms of aetiologically relevant factors and outcomes (childhood/adolescence and adulthood). There will be no exclusions based on context (e.g., inpatient vs.~outpatient) or design (RCT versus cohort).
Interventions
Psychological intervention will be any form of psychological treatment. Studies will not be excluded based on format of delivery (e.g., group-based, one-to-one). For this review, all patients included in an extracted sample must have received the psychological treatment. If there is anything to infer this is not the case, then the manuscript will be excluded. No exclusions will be made based on delivery method (face-to-face, teleconferencing, telephone, guided self-help).
Outcomes
Included studies will employ a measure of treatment effectiveness. This may include standardised and validated assessment tools (e.g., BDI-II) or standardised seizures measured (e.g., seizure frequency/intensity measures/Likert scales). Single item measures will be extracted if they are standarised and concern an index of seizure improvement/change. Measures of therapeutic process, change mechanism, or treatment satisfaction will not be included. Measurement may be clinician rated, informant rated or self-reported. The current review of effectiveness will not include health care cost or utilisation There will be no other exclusions based on what the effectiveness outcome/s are; they may cover focal seizure occurrence (reduction, remission, severity) psychological distress, dissociation, or peripheral measures of functioning (e.g., disability, quality of life). Outcomes may be expressed as either raw data, common effect-size metrics (r, d, OR) or as proportional change. Type of outcome measured will be coded for subgroup analysis. Psychological distress will include global distress measures and disorder specific measures (e.g., anxiety, depression, PTSD). If a study reports sub-scale scores for a given questionnaire (e.g., depression domain of the SCL-90) these will not be extracted.
The current review sought to establish effectiveness in the acute stage of treatment and durability of treatment at follow-up. Outcomes for the acute stage will be scores immediately following the end of treatment or (in the absence of post-treatment then) then the next nearest score that follows (must be below 5.9 months following treatment). Follow-up periods will include 0-6 month, 7-12-month, and 12-month+. If a study only reports study follow up data (e.g., 3 months following treatment) then this will be used for both analyses (pre-post, and then follow up).
Design
All study group designs will be included. Case studies and single case experimental design will be excluded. Case series that report quantitative outcome data will be included.
Extraction:
A standardised extraction sheet (separate file) and codebook (Appendix B) will be developed and pilot-tested with a sample of studies (k = 3). Our approach to missing study codes followed the approach of infer, initiate, impute (Pigott \& Polanin, 2020). Studies that remain with missing codes for moderator variables following contacting the authors will considered for multiple imputation approaches.
Data extraction will include coding of study characteristics, sample characteristics, outcome data, moderator variables and risk of bias indicators for all included studies. Study codes will allow for variables to be used in moderator analysis. Extraction will be conducted by the lead author while effect-size information will be coded by a second coder to mitigate errors. If coding disagreements arise then these will be resolved through review discussion between coders. Percentage agreement and inter-rater reliability statistics (Kappa {[}κ{]}, Cohen, 1960) will be used to quantify initial agreement between the first and second coder. Descriptive classifiers available for interpreting κ will be used (Landis \& Koch, 1977) consisting of slight (0-0.2), fair (0.2-0.4), moderate (0.4-0.6), substantial (0.6-0.8), and almost perfect (0.8-1.0). 25\% titles/abstracts and 100\% of full texts will be coded by a second coder.
Systematic reviews and meta-analyses should be fully transparent and reproducible (Pigott \& Polanin, 2020). To support this, we will pre-register the current protocol through OSF. Upon peer-review publication we will share review data, analytical syntax, review findings and version history through the first authors GitHub profile. In acquiring, processing, and reporting findings we will follow the PRISMA (2020) and the meta-analyses in psychotherapy (MAP-24; Fluckiger et al., 2018) guidelines. A table of studies and their key characteristics will be reported; if the total number of studies exceeds 50 then it will be placed in the supplementary material).
Data synthesis
Outcome data extracted from manuscripts will include N, premean, preSD, postmean, Pearson's r, effect-size (regardless of metric), and inferential statistics (i.e., ANOVA, regressions, p value) or proportion change (in the case of seizure freedom and improvement). Sample characteristics will include N and percentage rate of female, white, married and unemployed participants SES. In situations when N is reported but not \% then this will be calculated by the first coder. Treatment and study characteristics for extraction are described in Table 4.
When multiple independent samples are reported (e.g., CBT vs., psychoanalysis) for a given manuscript then both samples will be extracted. Studies with overlapping datasets will be excluded. Samples that use more robust approaches to missing data (e.g., ITT) will be preferred over approaches that do not (e.g., samples utilizing only data from patients with complete data).

Table 4:
Treatment and study characteristics for data extraction and moderation analysis
Treatment format Treatments will be coded as (i) individual, (ii) group, or (iii) mixed.

Treatment delivery Treatments will be coded as (i) face-to-face, (ii) telephone, (iii) video or (iv) blended (i.e., a combination).
Treatment modality Treatment type/modality code will be determined by author designation/description. We will not require any use of protocols or verification of techniques used. Treatment codes will be collapsed into the following domains: (i) behavioral, (ii) relational, (iii) cognitive-behavioural, (iv) psychoeducation, (v) body focused, (vi) eclectic/other, (vii) counselling unspecified. Samples that represented multiple treatments or eclectic treatments were also coded as other. Ambiguous treatment codes will be reviewed by three study authors (CG, GR, NP). Univariate moderator analysis will exclude the other treatment domain.

Treatment sessions The mean number of treatment sessions. If reported in hours then we will consider per hour to represent a treatment session. If mean number of sessions is not reported, we will extract maximum number of sessions in the study.
Treatment dosage Studies will be stratified into short (0-6 session/hours), medium (7-14) and long-term treatments (14+).
RoB rating Observational studies will be coded as `low, `moderate, `serious', `critical' or `no information.
RCTs will be coded as `low', `unclear', or `high' risk of bias.
Study location The country which the study is conducted in.
Publication year The year of print publication.
Control conditions Control conditions were coded as (i) no control, (ii) usual care or (iii) waiting list.
Study N During the analysis stage, samples will be stratified into sample size bins, including: (i) \textless25, (ii) 25-50, or (iii) 50+.
For study samples with multiple measures within a single outcome domain (e.g., two measure of psychological distress) all outcomes will be extracted. Reliability of coding (mean, SD, d) for effect-size data will be computed using a second coder in duplicate (all studies).
Seizure information
Seizure frequency reduction and remission will be extracted as separate dependent variables. The usual convention when reporting seizure frequency or remission is to report percentage reduction (i.e., proportion). We will record how studies define seizure reduction. In addition to \% reduction, any accompanying indicator of data average (mean, median) and distribution (e.g., SD, SE, IQR) will be extracted. If studies report the median and SE/IQR/range (i.e., instead of mean and SD) then these will be extracted, allowing the option for data transformation. If both are reported then both will be extracted.
Analysis
All analyses will be conducted using the R statistical analysis environment (R Core Team, 2020, v 4.0.2). Using a dummy data set (k = 5) the review team will develop analysis scripts prior to full data extraction. The outcome metrics for the study include (i) proportions (seizure frequency reduction, seizure remission), and (ii) standardised mean change (Becker, 1988; for all other outcomes).
Effect-size calculation.
For non-proportional or frequency based outcomes, Standardised mean change (Becker, 1988) will be calculated using the metafor package (Viechtbauer, 2020). This approach divides the pre-post mean change score by the pretreatment standard deviation and adjusting the standard error using the pre-post Pearson's r (Morris, 2008). For manuscripts that do not report the required data for effect-size calculation, authors will be contacted to request missing data. The exception to this will be for Pearson's r as it is so rarely reported. Instead, Pearson's r (when missing) will be imputed using an empirically derived estimate and then adjusted using sensitivity analyses.
If the required data remains unavailable two weeks after contacting authors, then a protocol will be followed to estimate the effect-size to avoid study exclusion (Appendix C). Aggregation of study samples, if required, will be conducted using the aggregate function of metafor using standard inverse-variance weighting of samples. Effect-sizes will be converted so that a positive value indicates improvement across time points. For studies that report significant effects, but no precise alpha level, d will be imputed using the smallest possible significance level possible for that sample size. Studies only reporting that there were no significant differences will be allocated d = 0 (Smith, 1980).
Adjusting for dependent outcomes.
As it is expected that several studies will report multiple outcomes and potentially multiple treatment comparisons, including all such effect-sizes within a standard meta-analysis would lead to violation to the assumption of statistical independence (Borenstein et al., 2021; Cheung 2019; Hoyt \& Del re, 2018). Two common approaches to this, (i) averaging across measures or (ii) selecting a single measure per study, both have limitations (Cheung 2019). Developing a measure/outcome preference is particularly unsuitable for this review of functional seizures as there is a distinct lack of consensus around the optimal outcomes to measure (Pick et al., 2020; Nicholson et al., 2020). Multilevel meta-analysis is a method that allows for inclusion of all measures, comparisons, and target problems, in non-aggregated form, while resolving the issue of statistical dependency (Noortgate, 2013). Multilevel meta-analyses will be performed using the metafor (Viechtbauer, 2020), dmetar (Harrer et al., 2019a), and meta (Schwarzer, 2020) packages, using available practical guidance (Harrer et al., 2019a). A three-level model will be used, including study effect-sizes (level-1), within-study variation (level-2) and between study variation (level-3). A random effects model will initially be used to calculate summary effect-size and variance estimates. Use of likelihood ratio tests will be used to determine significant within or between study differences in variance. Sensitivity analyses will include: (i) alternative values for imputation of r; (ii) removal of statistical outliers.
Since the original draft of this protocol, it came to the knowledge of the first author that an alternative analytical approach would be more suitable. While multi-level meta-analysis is an appropriate approach for statistical dependencies it is most well suited when dependencies have a predominantly hierarchical nature (e.g., samples with studies, studies within research teams). The current review was expected to have a greater emphasis upon correlational dependencies (i.e., multiple outcomes on identical samples). It has been proposed that robust variance estimation is a more suitable approach for this form of statistical dependency and therefore will be considered for analysis.
Subgroup and continuous variables will be used to attempt to explain variation between levels 2 and 3. If studies significantly differ in effect-size based on a variable/characteristic then this is a moderator. For categorical moderator analyses, a subgroup analysis will be conducted when at least 10 studies are eligible for inclusion and when there are at least two studies in each subgroup (Deeks et al., 2019). For continuous variables, a meta-regression will be conducted when at least six studies are eligible for inclusion in the specific meta-regression. Correction for multiple moderators (i.e., Bonferoni) will not be used due to low statistical power within moderator analyses. A separate three-level model will be used for studies that include follow-up data.
Due to expected high heterogeneity we will employ random-effects meta-analyses to estimate pooled and weighted effect-sizes (Higgins \& Green, 2006, Borenstein et al., 2021). 95\% confidence intervals will be calculated for included studies. Forest plots will be reported to visualise the overall pattern of the results if there is a feasible number of samples (k = \textless30, Pigott \& Polanin, 2020).
For samples that report seizure data as a proportion rather than a pre-post intervention change (i.e., \% seizure reduction/remission), then a separate random effects meta-analysis will be conducted focused on proportions (i.e, not SMD) using the metaprop R package (Viechtbauer, 2010). Prior to synthesis, individual study proportions will be transformed using the Freeman-Tukey (double arcsine) transformation, so that very low or high proportions are appropriately weighted (Barendregt et al., 2013). The transformation will adjust for variance constraints, as when proportions are close to the extremes (minimum of 0 or maximum of 1), variances are very small and will be given increased weightings when using the inverse of the variance to weight the studies. Proportions will be back-transformed and reported as percentages for interpretation of the pooled estimates.
Between-study heterogeneity will be assessed using I2 (Higgins \& Thompson, 2002) and the Q statistic (Cochran, 1954). I2 was interpreted as low (25-50\%), moderate (50-75\%) or high (75-100\%, Higgins et al., 2003). The impact of publication bias on treatment estimates will be visualised using funnel plots and assessed statistically using rank correlation tests (Begg \& Mazumdar, 1994), Egger's regression test for funnel plot asymmetry (Egger et al., 1997), and fail-safe N (Rosenthal method, Rosenthal, 1979).
Risk of bias and methodological quality
Bias has been described as the systematic error, or deviation from the truth within results (Boutron et al., 2021). Bias is not the same as methodological quality but has been described as a component of quality, in addition to generalizability and precision (Boutron et al., 2021). To varying degrees, bias may lead to inaccurate estimations of treatment effects. Tools assessing the risk of bias (RoB) for primary studies included in meta-analyses are widely available and commonly employed. Observational studies typically show higher RoB (i.e., poor internal validity) but greater representation of routine conditions (i.e., good external validity), whereas RCTs tend to show smaller RoB but are less representative of routine conditions. Based on prior reviews of psychological treatments in functional seizures (Perry et al., 2017), it is expected that the large proportion of eligible studies for the current review will be observational with high RoB.
The current review will use the ROB 2.0 for quality appraisal. This tool rates primary studies based on six criteria: (1) randomization procedure, (2) allocation concealment, (3) blinding of personnel, (4) blinding of outcome assessment, (5) incomplete outcome data, and (6) selective outcome reporting. This review will also include (7) treatment implementation which has been acknowledged as highly relevant for psychotherapy outcome research (Munder \& Barth, 2018). Each criterion is rated as `high', `low' and `unclear', with an overall study rating then being allocated. Finally, we will also use the functional seizure specific quality criteria used in a recent systematic review of functional seizures (Brown \& Reuber, 2016). Criteria include: (1) explicit reference to procedures that could allow PNES to be distinguished from an anxiety disorder; (2) comparability of PNES and epilepsy control groups in terms of age and gender; (3) consecutive recruitment of PNES group; (4) comparability of PNES and other control groups in terms of age and gender; (5) explicit reference to exclusion of PNES in other control groups; (6) adequately powered; (6) explicit reference to all PNES and epilepsy patients having video-EEG confirmed diagnoses; (7) explicit reference to exclusion of concurrent epilepsy in PNES group; (8) explicit reference to exclusion of PNES in epilepsy control group.
High risk of bias (studies or domains) is suspected to lead to over-estimation of treatment effect-sizes however within the field of psychotherapy there has been inconsistent findings reported (e.g., Cuijpers et al, 2010; Barth et al., 2013). Psychotherapy researchers are recommended to incorporate risk of bias results into meta-analytic analyses (Munder \& Barth, 2018; Boutron et al., 2021). This can be performed through (i) excluding studies based on quality, or (ii) assessing the degree to which risk of bias may moderate treatment outcome (Boutron et al., 2021). The current study will use the latter approach, comparing between risk of bias classifications.
The overall methodological quality of the evidence will be assessed by three reviewers using guidelines for the Grading of Recommendations, Assessment, Development and Evaluations (GRADE, Guyatt et al., 2008). This framework rates evidence quality for each meta-analytic outcome based on included study designs. Individual ratings are initially provided (high, moderate, low or very low) and are then downgraded (or upgraded) through evaluation of five separate criteria; risk of bias within included studies, inconsistencies in aggregated treatment effect, indirectness of evidence, imprecision and publication bias.

\hypertarget{applications}{%
\chapter{Applications}\label{applications}}

Some \emph{significant} applications are demonstrated in this chapter.

\hypertarget{example-one}{%
\section{Example one}\label{example-one}}

\hypertarget{example-two}{%
\section{Example two}\label{example-two}}

\hypertarget{final-words}{%
\chapter{Final Words}\label{final-words}}

We have finished a nice book.

  \bibliography{packages.bib}

\end{document}
